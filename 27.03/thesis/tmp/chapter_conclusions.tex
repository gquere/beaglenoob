\chapter{Conclusion: Linux on ARM vs other RISC architectures }

It is quite clear that beyond the similarities of the architecture (PWM, timers, interrupts), microcontrollers (PIC, AVR) and microprocessors (typically ARM) serve a different purpose. A microcontroller, though some can embed an tiny C OS, is a very simple machine, whereas ARMs are much more complex and can run a real OS, namely the Linux kernel.

The range of use of an ARM easily covers that of the microcontroller and extends it with powerful features such as the use of a camera. But doing the simplest of things such as blinking a LED at a very precise frequency can easily become an impossible to fulfill task in the user-space, and requires coding in the kernel. Developing in the kernel is a much tougher challenge and requires very advanced knowledge of the environment. A wild pointer would likely crash the system, and the try/fail/try/succeed approach would be very time consuming.
Therefore, appropriate considerations need to be taken into account when choosing the architecture of a system. What can be done on a microcontroller should probably be done on a microcontroller (an IR system, basic use of sensors, servos), and what requires a more powerful system (webserver with database, logins, file operations, cameras ...) should lead to using an ARM (in conjunction to a DSP for audio/video usually). It sums up to this rule of thumb :
"There are hardware mistakes that no amount of software can patch".

In my opinion, the best approach to a complex system consisting partly of real-time software is an hybrid ARM/PIC, with an i2c bus for the communication. The distinction between the (possible) user-interface and the real-time part of the system would be made clear, and the only drawback I can really think of is that the real-time part would receive orders with a slight delay.

%\begin{thebibliography}{50}
%\bibitem {bbgts} Getting started: \url{http://beagleboard.org/static/beaglebone/a3/README.htm}
%\bibitem {bbsrm} SRM:\url{http://beagleboard.org/static/beaglebone/a3/Docs/Hardware/BONE\_SRM.pdf}
%\bibitem [agimg]Official image: \url{http://www.angstrom-distribution.org/demo/beaglebone/}
%\bibitem [amtrm]AM3359 TRM: \url{http://www.ti.com/product/am3359}
%\bibitem [putty]Putty: \url{http://www.chiark.greenend.org.uk/~sgtatham/putty/download.html}
%\end{thebibliography}
